\section{Existing General Purpose Platforms} 

Today's general purpose data platforms fall short of supporting secure, real-time decisions on live data. To illustrate this point, consider a movie recommendation engine as shown in Figure~\ref{fig:related}.

\0
\begin{figure}[h]
  \center{
 \includegraphics[scale=0.5]{figures/existing-solutions}
  }
  \vskip -.15in
  \caption{\small{The implementation of a recommendation system.}}
\label{fig:related}
\end{figure}


Figure 3: A Movie recommendation engine example.

Users logs (e.g., viewing preference, ratings) are collected via a message broker such as Kafka and stored in a distributed files system, such as HDFS or S3. A Hadoop or Spark job will periodically read these logs and train a model to capture user recommendations. Then, these models are pushed into a key value store, such as HBASE or DynamoDB, from which they are served to users. For instance, when a user logs in, the application will query the key-value store to retrieve the movie recommendations for that user. While such a key-value store can provide millisecond-level delays, this solution has several significant drawbacks:

{\bf Simple decisions}: With a simple key-value store it is hard to support sophisticated decisions, such as ML classification or clustering. One way to support such decisions would be to pre-compute all possible decisions. Unfortunately, in general, this is infeasible due to ``the curse of dimensionality''. Another possibility would be to run non-trivial algorithms on non-trivial amounts of data. Unfortunately, this will compromise the query latency.  

{\bf Slow updates}: Typically, the intermediate representation (e.g., model) is updated hourly, or even daily. While Spark increases the update frequency it is still in the order of hours, as fundamentally, once need to read data from the distributed file system, re-compute the model and push the changes to the key-value store.

{\bf Weak security}: In most existing solution, security reduce to encryption at rest or encryption over network. The pre-processing and decision logic are vulnerable to any 3rd party attacks. This is in particular troublesome when computations run on the public clouds, which is fast becoming the norm, rather than the exception. 

In the research community, two techniques are emerging as possible solutions to protect the application's integrity and user's privacy. The first is computing on encrypted data which has been popularized by systems such as CryptDB and Mylar. The second is leveraging hardware enclaves, which are now integrated in both Intel and ARM chips to securely run arbitrary software in untrusted environments, such as the public cloud. However, none of these techniques are perfect. While computing on encrypted data makes minimal assumption about the hosted environment, it limits the kind of computations one can run. In contrast, hardware enclaves allow one to run arbitrary code, but they incur non-trivial overhead.  
